% !TEX encoding = UTF-8
%======中文摘要内容格式:{中文摘要}{关键词}
\ZhAbstract{湖南师范大学创建于1938年,位于历史文化名城长沙,是国家“211工程”重点建设的大学,国家“双一流”建设高校,教育部与湖南省重点共建“双一流”建设高校,教育部普通高等学校本科教学工作水平评估优秀高校,湖南省“世界一流学科建设高校”。学校现有7个校区,占地2744余亩,建筑面积125余万平方米。主校区西偎麓山,东濒湘江,风光秀丽,是全国绿化“400佳”单位之一。

学校设有24个学院,现招生本科专业81个,本科和研究生教育覆盖哲学、经济学、法学、教育学、文学、历史学、理学、工学、医学、管理学、艺术学等11大学科门类。学校拥有伦理学、英语语言文学、中国近现代史、发育生物学、理论物理、基础数学等6个国家重点学科,“语言与文化”学科群主建学科外国语言文学入选国家“世界一流”建设学科,教育学、数学、哲学、中国语言文学、生物学5个学科入选湖南省“国内一流建设学科”,法学、马克思主义理论、体育学、新闻传播学、物理学、化学、地理学、音乐与舞蹈学、美术学、政治学、心理学、中国史、生态学、理论经济学、统计学等15个学科入选湖南省“国内一流培育学科”;化学、临床医学2个学科进入ESI前1\%;拥有21个博士学位授权一级学科、教育博士专业学位授权类别、37个硕士学位授权一级学科和24种硕士专业学位授权类别,以及18个博士后科研流动站;拥有中国特色社会主义道德文化省部共建协同创新中心,拥有教育部人文社会科学重点研究基地湖南师范大学道德文化研究中心、国家体育总局体育社会科学重点研究中心;拥有中国语言文学、历史学国家文科基础学科人才培养和科学研究基地、国家生命科学与技术人才培养基地、全国大学生文化素质教育基地、教育部基础教育课程研究湖南师范大学中心、国家红色经典艺术教育示范基地、国家卓越法律人才教育培养基地、国家卓越中学教师培养计划实施院校、国家卓越医生教育培养计划项目试点高校、中华优秀传统文化传承基地(花鼓戏)、教育部体育美育浸润行动计划入选高校等16个国家级人才培养和科学研究基地(中心);拥有省部共建淡水鱼类发育生物学国家重点实验室,石化新材料与资源精细利用、动物多肽药物创制2个国家地方联合工程实验室,2个省部共建国家重点实验室培育基地,4个省部共建教育部重点实验室,拥有多倍体鱼繁殖与育种技术教育部工程研究中心、农业部鲤鲫遗传育种中心;拥有1个省部共建“2011协同创新中心”、3个湖南省“2011协同创新中心”、拥有生物科学和中国语言文学2个湖南省基础学科拔尖学生培养基地。
}{湖南师范大学,211重点大学,双一流建设高校}
%======中文摘要内容格式:{英文摘要}{关键词}
\EnAbstract{Founded in 1938, Hunan Normal University (HUNNU) enjoys a history of 80 years. It is also one of the leading universities building under the national “211 project” and “Double First-Class Strategic Plan”.

Hunan Normal University puts stress on fundamental and quality education, while placing equal stress on the cultivation of both application-oriented talents and research experts. Up to 2017, it has cultivated over 400,000 students. Hunan Normal University enjoys a high reputation in society for its excellent quality education. It now boasts 3 CFCRS (Chinese-foreign Cooperation in Running Schools) programs and 185 partner institutions in 45 countries and has co-established 3 Confucius Institutes in Russia, Korea and the U.S. respectively.
}{HUNNU, 211 project, Double First-Class Strategic Plan}