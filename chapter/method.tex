% !TEX encoding = UTF-8
\chapter{材料与方法}
% \Method

正确编译需要以下几个部分(这是一个列表环境):
\begin{itemize}
    \item 一个基本的\TeX{}发行版
    \item CJK或XeCJK(供\LaTeX{})宏包
    \item ctex宏包(提供ctexbook文档).
    \item 中文字体
    \item 如果要使用 biblatex 进行文献列表和引用的排版的话, 还需要 biblatex 宏包。
\end{itemize}

\section{模板使用}
\subsection{模板文件结构\label{sec:files}}
整个模板根目录的文件列表如下:
\begin{center}
    \begin{table}
        \setlength{\tabcolsep}{10mm}{
            \begin{tabularx}{15cm}{llc}
                \toprule
                文件                     & 说明                   & 备注                 \\
                \midrule
                WMUDoctor.cls            & WMUDoctor宏包          & \textcolor{red}{{*}} \\
                WMU.cfg                  & WMU宏包配置文件        & \textcolor{red}{{*}} \\
                WMUbib.bst               & 引文样式文件           & \textcolor{red}{{*}} \\
                references/reference.bib & bib数据库              & \textcolor{red}{{*}} \\
                figures/wmu.jpg          & 温州医科大学校名标准字 & \textcolor{red}{{*}} \\
                WMUBachelorTemplate.tex  & \TeX{}样例文件         & \textcolor{red}{{*}} \\
                \bottomrule
            \end{tabularx}}
    \end{table}
\end{center}
注: \textcolor{red}{{*}} 表示\LaTeX{}模板必须的文件。
\subsection{示例}
对于论文中最常使用的一些功能在本节中给出示例。
\subsection{公式}
\begin{equation}
    \hat{H}=\frac{\epsilon}{2}\hat{\sigma}_{z}-\frac{\Delta}{2}\hat{\sigma}_{x}+\sum_{k}\omega_{k}\hat{b}_{k}^{\dagger}\hat{b}_{k}+\sum_{k}\frac{g_{k}}{2}\hat{\sigma}_{z}(\hat{b}_{k}+\hat{b}_{k}^{\dagger})\label{eq:sbm}
\end{equation}

根据公式\ref{eq:sbm}可知,这个是对公示的引用。
\begin{align}\label{eqn3:9}
    \int_{-\infty}^{+\infty}S(\tau,f)\,\mathrm{d}\tau & =\int_{-\infty}^{+\infty}x(t)\left\{\int_{-\infty}^{+\infty}\frac{|f|}{\sqrt{2\pi}}e^{\frac{-|f|^2(\tau-t)^2}{2}}\,\mathrm{d}\tau\right\}e^{-j2\pi ft}\,\mathrm{d}t\notag                                  \\
                                                      & =\int_{-\infty}^{+\infty}x(t)e^{-j2\pi ft}\left\{\int_{-\infty}^{+\infty}\frac{1}{\sqrt{\pi}}e^{-\left[\frac{|f|(\tau-t)}{\sqrt{2}}\right]^2}\,\mathrm{d}\frac{|f|(\tau-t)}{\sqrt{2}}\right\}\,\mathrm{d}t
\end{align}
令$\theta=\frac{|f|(\tau-t)}{\sqrt{2}}$,则式\eqref{eqn3:9}可改写为
\begin{align}\label{eqn3:10}
    \int_{-\infty}^{+\infty}S(\tau,f)\,\mathrm{d}\tau & =\int_{-\infty}^{+\infty}x(t)e^{-j2\pi ft}\,\mathrm{d}t\frac{1}{\sqrt{\pi}}\int_{-\infty}^{+\infty}e^{-\theta^2}\,\mathrm{d}\theta\notag \\
                                                      & =\int_{-\infty}^{+\infty}x(t)e^{-j2\pi ft}\,\mathrm{d}t\frac{2}{\sqrt{\pi}}\int_{0}^{+\infty}e^{-\theta^2}\,\mathrm{d}\theta\notag       \\
                                                      & =\int_{-\infty}^{+\infty}x(t)e^{-j2\pi ft}\,\mathrm{d}t\notag                                                                            \\
                                                      & =X(f)
\end{align}
\subsection{表格}
\begin{table}[H]
    \begin{center}
        \caption{希腊字母表\label{tab:Greek}}
        \setlength{\tabcolsep}{10mm}{
        \begin{tabularx}{15cm}{ccccc}
            \toprule[1.5pt]
            Alpha    & Beta    & Gamma    & Delta    & Theta    \\
            \midrule[0.75pt]
            $\alpha$ & $\beta$ & $\gamma$ & $\delta$ & $\theta$ \\
            $A$      & $B$     & $\Gamma$ & $\Delta$ & $\Theta$ \\
            \bottomrule[1.5pt]
        \end{tabularx}}
    \end{center}
\end{table}
这是对表\ref{tab:Greek}的引用

\begin{table}[htbp]
    \caption{不同电力系统频率测量算法时间复杂度比较}\label{table2:1}
    \vspace{0.5em}\centering\zihao{5}
    \setlength{\tabcolsep}{4mm}{
    \begin{tabularx}{15cm}{cccc}
        \toprule
        算法     & 加法                  & 乘法                      & 时间复杂度    \\
        \midrule
        TQDS     & $QN^2+QN/2+Q+1$       & $QN^2$                    & $O(N^2)$      \\
        WIFFT    & $(QN+1)\log_2(QN+1)$  & $(QN+1)*(1+\log_2(QN+1))$ & $O(N\log_2N)$ \\
        本章算法 & $3(QN+1)\log_2(QN+1)$ & $(QN+1)(1+3\log_2(QN+1))$ & $O(N\log_2N)$ \\
        \bottomrule
    \end{tabularx}}
    \vspace{\baselineskip}
\end{table}

本章对时域准同步算法(Time Domain Quasi-synchronous,TQDS)、加窗插值~FFT~算法(Windowed Interpolated FFT,WIFFT)以及本章所提算法的时间复杂度进行分析。因~TQDS~需要进行迭代运算,故设总采样点数为~$QN+1$,其中~$Q$~为迭代次数,$N$~为单次迭代所需的数据点长度。TQDS~共需要~$QN^2$~次加法和~$QN^2+QN/2+Q+1$~次乘法,因此~TQDS~的时间复杂度为~$O(N^2)$。WIFFT~的计算量主要为~FFT~运算,共需进行~$(QN+1)\log_2(QN+1)$~次加法和~$(QN+1)(1+\log_2(QN+1))$~次乘法,因此~WIFFT~的时间复杂度为~$O(N\log_2N)$。对于本章所提出的算法,由于线性卷积运算采用快速卷积来进行计算,因此共需进行~$3(QN+1)\log_2(QN+1)$~加法和~$(QN+1)(1+3\log_2(QN+1))$~次乘法,算法时间复杂度为~$O(N\log_2N)$。表~\ref{table2:1}~对三种频率测量算法的时间复杂度进行了对比。由表~\ref{table2:1}~可见,TQDS~的时间复杂度比其它两种算法要高,本章算法和~WIFFT~时间复杂度相当,有利于算法的实时实现。
\subsection{图形}
这个示例为插入图片:
\begin{figure}[H]
    \centering
    \includegraphics[width=0.8\textwidth]{f1.jpg}%图片名称,放在/figures目录下
    \caption{图片插入\label{fig:fig}}
\end{figure}

具体代码:
\begin{verbatim}
%抄写环境
\begin{figure}[H]
\centering
\includegraphics[width=0.8\textwidth]{f1.jpg}%图片放在/figures目录下
\caption{图片插入\label{fig:fig}}
\end{figure}
\end{verbatim}
\begin{figure}[H]
    \centering
    \includegraphics[width=0.8\textwidth]{WMU.jpg}
    \caption{温州医科大学题字及LOGO\label{fig:WMU}}
\end{figure}
对于图\ref{fig:fig}和图\ref{fig:WMU}的引用。


\subsection{引用}
\subsubsection{交叉引用}
对所有需要引用的公式、表格、图形,执行插入--标签后,即可使用插入-- 交叉引用自动产生引用。
\begin{itemize}
    \item 哈密顿量见方程~\eqref{eq:sbm}。
    \item 希腊字母表见表~\ref{tab:Greek}。引用格式与方程引用格式不同
    \item 校名标准字如图~\ref{fig:WMU}。 引用格式与方程引用格式不同
\end{itemize}
具体见代码:
\begin{verbatim}
\begin{itemize}
\item 哈密顿量见方程~\eqref{eq:sbm}。
\item 希腊字母表见表~\ref{tab:Greek}。引用格式与方程引用格式不同
\item 校名标准字如图~\ref{fig:WMU}。 引用格式与方程引用格式不同
\end{itemize}
\end{verbatim}

\subsubsection{文献引用}
将引文的bib数据库(默认文件名为reference.bib)放入模板根目录下的references文件夹,即可通过插入--文献引用自动产生引文。
\begin{itemize}
    \item Journal:An article \upcite{ELIDRISSI94,MELLINGER96,SHELL02,cnarticle}。
    \item Book:An book \cite{IEEE-1363,tex,companion}。
    \item Conference:A conference \cite{kocher99,DPMG,cnproceed}。
    \item Manual:A manual\cite{NPB2}.
    \item MasterThesis:\cite{zhubajie,metamori2004,shaheshang,FistSystem01}.
\end{itemize}
\section{伪代码实现}
\begin{algorithm}
    \caption{放进冰箱的大象}\label{算法实例}
    \begin{algorithmic}
        \REQUIRE 有一只大象
        \ENSURE 放进冰箱里
        \FOR {没有剩余的大象}
        \IF {大象比冰箱大}
        \STATE 把大象分割
        \ENDIF
        \ENDFOR
        \STATE 第一步
        \STATE 第二步
        \STATE 第三步
    \end{algorithmic}
    AAA\end{algorithm}
\subsection{代码展示}
可以把你的程序添加到附录里,展示自己的工作。
\begin{lstlisting}[language={[ANSI]C}, numbers=left]
#include <stdio.h>
int main(int argc, char ** argv)
{
/*打印Hello,world*/
printf("Hello, world!\n");

return 0;
}
\end{lstlisting}
\section{依赖}
WMUDoctor依赖于以下宏包,这些宏包在常见的\LaTeX{}发行版中都包括,在安装使用之前,请确定你的\TeX{}发行版中都已正常安装这些宏包
\begin{table}[H]
    \centering
    \setlength{\tabcolsep}{11mm}{
    \begin{tabularx}{15cm}{cccc}
        \toprule
        \multicolumn{4}{c}{依赖宏包} \\
        \midrule
        {footmisc}    & {amsmath}  & {amsfonts} & {amssymb}   \\
        {graphicx}    & {svgnames} & {xcolor}   & {mathptmx}  \\
        {float}       & {fontenc}  & {fancyhdr} & {lastpage}  \\
        {etoolbox}    & {fancy}    & {caption}  & {array}     \\
        {makecell}    & {forloop}  & {xstring}  & {hyperref}  \\
        {tabularx}    & {enumitem} & {ntheorem} & {algorithm} \\
        {algorithmic} & {bibentry} & {xeCJK}    & {CJK}       \\
        {listings}    & {courier}  & {}         & {}          \\
        \bottomrule
    \end{tabularx}}
\end{table}
如果你尚未安装这些宏包,可以启动你的 \TeX{} 发行版的宏包管理器
来安装;或者到 \url{http://www.ctan.org} 上搜索下载并安装。

\section{基本设置}

\begin{enumerate}[label=(\arabic*)]
    \item 图片搜索路径默认设置为模板根目录下的figures/。
    \item bib数据库默认设置为模板根目录下的references/reference.bib。其中bib文件可由任意文献库管理软件自动生成。
\end{enumerate}

% 简单帮助
\section{文字命令}
\LaTeX 提供了一系列命令,用于修改字体、字号、数字等的呈现形式。

本论文中字体如下:
\subsection{字体}
\begin{verbatim}
宋体: \songti    启用宋体。
黑体: \heiti     启用黑体。
仿宋: \fangsong  启用仿宋。
楷书: \kaishu    启用楷书。
\end{verbatim}
{\songti 宋体} {\heiti 黑体}    {\fangsong 仿宋}     {\kaishu 楷书}

\subsection{字形}
\begin{verbatim}
粗体BOLD:  \textbf{粗体BOLD} 启用粗体
斜体ITALIC:\textbf{粗体BOLD} 启用斜体
\end{verbatim}
{\textbf{粗体BOLD}} {\textit{斜体ITALIC}}

\subsection{字号}%
\begin{center}
    \setlength{\tabcolsep}{5.5mm}{
	\begin{tabularx}{15cm}{cccccccc}
		\toprule
		初号 & 小初 & 一号 & 小一 & 二号 & 小二 & 三号 & 小三 \\
		0 & -0 & 1 & -1 & 2 & -2 & 3 & -3 \\
		\hline
		四号 & 小四 & 五号 & 小五 & 六号 & 小六 & 七号 & 八号 \\
		4 & -4 & 5 & -5 & 6 & -6 & 7 & 8 \\
		\bottomrule
	\end{tabularx}}
\end{center}

\clearpage
\begin{tabular}{ll}
    {\zihao{0}初号}  & {\zihao{0}42}   \\
    {\zihao{-0}小初} & {\zihao{-0}36}   \\
    {\zihao{1}一号}  & {\zihao{1}26}   \\
    {\zihao{-1}小一} & {\zihao{-1}24}   \\
    {\zihao{2}二号}  & {\zihao{2}22}   \\
    {\zihao{-2}小二} & {\zihao{-2}18}   \\
    {\zihao{3}三号}  & {\zihao{3}16}   \\
    {\zihao{-3}小三} & {\zihao{-3}15}   \\
    {\zihao{4}四号}  & {\zihao{4}14}   \\
    {\zihao{-4}小四} & {\zihao{-4}12}   \\
    {\zihao{5}五号}  & {\zihao{5}10.5} \\
    {\zihao{-5}小五} & {\zihao{-5}9}    \\
    {\zihao{6}六号}  & {\zihao{6}7.5}  \\
    {\zihao{-6}小六} & {\zihao{-6}6.5}  \\
    {\zihao{7}七号}  & {\zihao{7}5.5}  \\
    {\zihao{8}八号} & {\zihao{8}5}    \\
\end{tabular}

\subsection{划线标记}
\begin{verbatim}
下划线:    \uline{下划线}      启用下划线
双下划线:  \uuline{双下划线}   启用双下划线
波浪线:    \uwave{波浪线}      启用波浪线
删除线:    \sout{删除线}       启用删除线
斜删除线:  \xout{斜删除线}     启用斜删除线
\end{verbatim}
{\uline{下划线}} {\uuline{双下划线}} {\uwave{波浪线}} {\sout{删除线}} {\xout{斜删除线}}

\clearpage
