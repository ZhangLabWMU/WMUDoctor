% !TEX TS-program = xelatex
% !TEX encoding = UTF-8
\documentclass{HUNNUthesis}
\graphicspath{{figures/}}
\begin{document}
%=====封皮页需要自己填写的内容
% 标题样式 使用 \title{{}}; 使用时必须保证至少两个外侧括号
%  如:短标题 \title{{第一行}},
%      长标题 \title{{第一行}{第二行}}
%      超长标题\tiitle{{第一行}{...}{第N行}}
\title{{湖南师范大学本科生毕业设计模板}}%中文标题
% 标题样式 使用 \entitle{{}}; 使用时必须保证至少两个外侧括号
%  如:短标题 \entitle{{First row}},
%      长标题 \entitle{{First row}{ Second row}}
%      超长标题\entitle{{First row}{...}{ Next N row}}
% 注意:英文标题多行时,需要在开头加个空格,防止摘要标题处英语单词粘连。
\entitle{{The Template of HUNNU Design}{for undergraduate}}%英文标题
\author{Angus Zhang}%论文作者
\major{重症医学专业}%专业、年级
\advisor{李老师}%指导老师
\college{工程与设计学院}%学院
\serialnumber{182001010}%学号
\code{080202}%专业代码
\submityear{二〇}%提交年份二〇二〇就填二〇
\submitmonth{二}%提交月份
%%======生成封面(几乎不需要更改)
\maketitle
\frontmatter
%======生成目录
\tableofcontents
%======插图索引和附表索引
\renewcommand{\listfigurename}{插图索引}
\renewcommand{\listtablename}{附表索引}
\setcounter{lofdepth}{1}
\newcommand{\loflabel}{图~}
\renewcommand{\numberline}[1]{\songti\zihao{4}\loflabel~#1\hspace*{\baselineskip}}
\addcontentsline{toc}{chapter}{插图索引}
\listoffigures
\newcommand{\lotlabel}{表~}
\renewcommand{\numberline}[1]{\songti\zihao{4}\lotlabel~#1\hspace*{\baselineskip}}
\addcontentsline{toc}{chapter}{附表索引}
\listoftables
%% !TEX encoding = UTF-8
%======中文摘要内容格式:{中文摘要}{关键词}
\ZhAbstract{1}{2}{3}{4}{温州医科大学;浙江省重点建设高校}
%======中文摘要内容格式:{英文摘要}{关键词}
\EnAbstract{1}{2}{3}{4}{WMU, Priority Development University of Zhejiang Province}%学校模板貌似把摘要和正文一起编页码了,所以这个部分挪到了后面
%======文章主体
\mainmatter
% !TEX encoding = UTF-8
%======中文摘要内容格式:{中文摘要}{关键词}
\ZhAbstract{1}{2}{3}{4}{温州医科大学;浙江省重点建设高校}
%======中文摘要内容格式:{英文摘要}{关键词}
\EnAbstract{1}{2}{3}{4}{WMU, Priority Development University of Zhejiang Province}%如果中英文摘要和目录一起罗马数字编页码,请把部分移到\mainmatter前
\include{chapter/chapter1}%各个章节内容单独分开撰写,方便编译和组织;
% !TEX encoding = UTF-8
\chapter{我是一级标题}
我是正文内容,我是正文内容,我是正文内容,我是正文内容,我是正文内容,我是正文内容,我是正文内容,我是正文内容,我是正文内容,我是正文内容,我是正文内容,我是正文内容,我是正文内容,我是正文内容,我是正文内容,我是正文内容,我是正文内容,我是正文内容,我是正文内容,我是正文内容,我是正文内容,我是正文内容,我是正文内容,我是正文内容,我是正文内容,我是正文内容,我是正文内容,我是正文内容,我是正文内容,我是正文内容,我是正文内容,我是正文内容,我是正文内容,我是正文内容,我是正文内容,我是正文内容,我是正文内容,我是正文内容,我是正文内容,我是正文内容,我是正文内容,我是正文内容,我是正文内容,我是正文内容,我是正文内容,我是正文内容,我是正文内容,我是正文内容,我是正文内容,我是正文内容,我是正文内容,我是正文内容,我是正文内容,我是正文内容,我是正文内容,我是正文内容,我是正文内容,我是正文内容,我是正文内容,我是正文内容,我是正文内容,我是正文内容,我是正文内容,我是正文内容,我是正文内容,我是正文内容,我是正文内容,我是正文内容,我是正文内容,我是正文内容,我是正文内容,我是正文内容,我是正文内容,我是正文内容,我是正文内容,我是正文内容,我是正文内容,我是正文内容,我是正文内容,我是正文内容,我是正文内容,我是正文内容,我是正文内容,我是正文内容,我是正文内容,我是正文内容,我是正文内容,我是正文内容,我是正文内容,我是正文内容,我是正文内容,我是正文内容,我是正文内容,我是正文内容,我是正文内容,我是正文内容,我是正文内容,我是正文内容,我是正文内容,我是正文内容,我是正文内容,我是正文内容,我是正文内容,我是正文内容,我是正文内容,我是正文内容,我是正文内容,我是正文内容,我是正文内容,我是正文内容,我是正文内容,我是正文内容,我是正文内容,我是正文内容,我是正文内容,我是正文内容,我是正文内容,我是正文内容,我是正文内容,我是正文内容,我是正文内容,
\section{我是二级标题}
\subsection{我是三级标题}
\include{chapter/chapter3}
\include{chapter/chapter4}
\include{chapter/chapter5}
%=======论文后部
\backmatter
%=======参考文献
\bibdatabase{references/reference}%参考文献数据库
\printbib
% !TEX encoding = UTF-8
\Appendix
这里是附录页,附上你的程序或必要的相关知识

\bf\heiti\color{red}{若要生成目录和参考文献的编译方式:} \color{black}XeLaTeX -->BibTeX --> XeLaTeX--> XeLaTeX%附录
% !TEX encoding = UTF-8
\chapter{致~~~~谢}

我是致谢。我是致谢。我是致谢。我是致谢。我是致谢。

\textit{我是致谢。我是致谢。我是致谢。我是致谢。我是致谢。}

\textbf{我是致谢。我是致谢。我是致谢。我是致谢。我是致谢。}

\textbf{\textit{我是致谢。我是致谢。我是致谢。我是致谢。我是致谢。}}

I'm thanks. I'm thanks. I'm thanks. I'm thanks. I'm thanks.

\textit{I'm thanks. I'm thanks. I'm thanks. I'm thanks. I'm thanks.}

\textbf{I'm thanks. I'm thanks. I'm thanks. I'm thanks. I'm thanks.}

\textbf{\textit{I'm thanks. I'm thanks. I'm thanks. I'm thanks. I'm thanks.}}

{\heiti 你好} {\heiti{\bfseries 你好}}
\clearpage%致谢
\end{document}